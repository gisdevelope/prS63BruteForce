
\begin{DoxyEnumerate}
\item При запуске клиент выполняет поиск сервера или этот поиск может быть выполнен вручную.
\item Если сервер найден, то клиент запрашивает сервер о подключении (connection\+Request);
\item Сервер подтверждает подключение (connection\+Confirm) и передает клиенту размер блока для подбора ключей и первые 8 байт которые нужно расшифровывать;
\item Получив подтверждение клиент переходит в режим ожидания команд от сервера;
\item Раз в фиксированный промежуток времени клиент проверяет подключенность к серверу (state\+Request). Если за этот промежуток времени уже было подключение к серверу для передачи данных, то запрос не выполняется, а время ожидания начинается после завершения обмена данными.
\item Находясь в режиме ожидания, клиент получает от сервера начальный ключ блока данных (transfer\+Data).
\item Клиент подтверждает (accept\+Data) или отказывается (refuse\+Data) от приема данных. При отказе, сервер запрашивает состояние клиента и если клиент находится в режиме ожидания, то через определённый промежуток времени повторяет передачу начального ключа, при этом ключ может отличааться от первоначального.
\item Получив ключ выполняет подбор пароля для первых 8 байт и если в результате дешиврации получается начальная сигнатура zip-\/файла, то передает этот ключ серверу (transfer\+Data). Сервер подтверждает (accept\+Data) получения ключа. При отказе (refuse\+Data) клиент через определенный промежуток времени повторяет передачу данных до тех пол, пока данные не будут приняты сервером. Если в результате подбора полученно несколько ключей, то они передаются последовательно;
\item После проверки всех ключей блока, клиент запрашивает новый блок ключей (transfer\+Request) и либо получает его или переходит в режим ожидания.
\item В любой момент может быть получен запрос состояния от сервера (state\+Request). В ответ на этот запрос клиент возвращает своё состояние (state\+Confirm). 
\end{DoxyEnumerate}